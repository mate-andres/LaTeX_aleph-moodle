\documentclass[10pt]{article}

\usepackage[utf8]{inputenc}
\usepackage[T1]{fontenc}

\usepackage[spanish]{babel}

\usepackage{aleph-comandos}
\usepackage{aleph-moodle}

% -- Paquetes adicionales
\usepackage{amssymb}
\usepackage{mathpazo}

\usepackage[margin=2cm]{geometry}

\begin{document}

%%%%%%%%%%%%%%%%%%%%%%%%%%%%%%%%%%%%%%%%%%%%%%%%%%
\begin{quiz}{Prueba01}
%%%%%%%%%%%%%%%%%%%%%%%%%%%%%%%%%%%%%%%%%%%%%%%%%%

%%%%%%%%%%%%%%%%%%%%%%%%%%%%%%%%%%%%%%%%%%%%%%%%%%
%%%%%%%%%%%%%%%%%%%%%%%%%%%%%%%%%%%%%%%%%%%%%%%%%%
%%%%%%%%%%%%%%%%%%%%%%%%%%%%%%%%%%%%%%%%%%%%%%%%%%
\begin{multi}{preg01}
    Dada la transformaci\'on lineal
    \[
        \funcion{T}{\R^2}{\R^2}{x}{(2x_1,x_1+x_2).}
    \]
    El vector que pertenece al n\'ucleo de $T$ es: $\arcsen(x)$
    \item $(1,-1)$
    \item $(2,-2)$
    \item $(0,1)$
    \item* $(0,0)$
\end{multi}

%%%%%%%%%%%%%%%%%%%%%%%%%%%%%%%%%%%%%%%%%%%%%%%%%%
%%%%%%%%%%%%%%%%%%%%%%%%%%%%%%%%%%%%%%%%%%%%%%%%%%
%%%%%%%%%%%%%%%%%%%%%%%%%%%%%%%%%%%%%%%%%%%%%%%%%%
\begin{multi}{preg02}
    Dada la matriz 
    \[
        A=
        \begin{pmatrix}
        1&3\cr
        5&7
        \end{pmatrix}.
    \]
    El determinante de $A$ es:
    \item $5$
    \item* $-8$
    \item $3$
    \item $0$
\end{multi}

\end{quiz}

\end{document}
